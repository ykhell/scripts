\documentclass[a4paper]{report}
\usepackage{../template}
\begin{document}
\section{Strukturtheorie zu Gruppen (``Einige Aussagen'')}
Sei im Weiteren $M$ ein Monoid, $G$ eine Gruppe und $X$ eine Menge.
\addcontentsline{toc}{section}{Wirkung}
\begin{defi}[\textbf{Wirkung}]
\label{def:Wirkung}
  Eine Abbildung
  $$\lambda : M \times X \to X, (m,x) \mapsto m \cdot x := \lambda(m,x)$$
  heißt Linkswirkung (left action, Linksoperation) von $M$ auf $X$, wenn es gelten $\forall x \in X, m, m' \in M:$
  \begin{enumerate}[(i)]
    \item Neutrales Element: $e\cdot x = x$
    \item Assoziativität: $m\cdot (m'\cdot x) = (m\cdot m') \cdot x$
  \end{enumerate}
\end{defi}
\begin{bez*}
  Ist $M$ eine Gruppe, so heißt $\lambda$ auch Gruppenwirkung und $X$ heißt Links-$M$-Menge.
\end{bez*}
\begin{bem*} Analog kann man auch Rechtswirkungen
  $$\rho : X \times M \to X, (x,m) \mapsto x \cdot m$$
  definieren. (Axiome: $x\cdot e = c$ und $(x\cdot m)\cdot m' = x \cdot (m \cdot m')$)
\end{bem*}
\begin{bem*}[Übung]
  Jede Links-$G$-Wirkung kann man in eine Rechts-$G$-Wirkung überführen:
  zu $\lambda: G \times X \to X$ definiere $\rho : X \times G \to X$ durch
  $$\rho(x,g) := \lambda(g^{-1}, x) \iff x \cdot g := g^{-1} \cdot x$$
\end{bem*}
\begin{prop}[Alternative Beschreibung von Wirkungen] \item
\begin{enumerate}[(a)]
  \item Sei $\lambda: G \times X \to X$ eine Linkswirkung, dann ist
        $$\ph : G \to \Bij(X), g \mapsto (\ph_{g}: X \to X, x \mapsto gx)$$
        ein wohl-definierter Gruppenhomomorphismus.
  \item Sei $\ph : G \to \Bij(X)$ ein Gruppenhomomorphismus, dann ist
        $$\lambda: G \times X \to X, (g,x) \mapsto \ph(g)(x)$$
        eine Linkswirkung von $G$ auf $X$.
\end{enumerate}
\end{prop}
\begin{proof}[Beweis]
  \begin{enumerate}[(a)]
          \item Für $g \in G$ sei $\ph_{g} : X \to X, x \mapsto gx$, dann gelten:
          $\ph_{e}: X \to X, x \mapsto ex = x$ ist $\id_{X}$ (Axiom (i)), und
          $$(*) \quad \ph_{g} \circ \ph_{g'} = \ph_{gg'}$$
          denn $\forall x \in X:$
          $$(\ph_{g} \circ \ph_{g'})(x) = \ph_{g}(\ph_{g'}(x)) = g(g'x) \overset{(ii)} = (gg')x = \ph_{gg'}(x)$$
          Damit folgen:
          \begin{enumerate}[1.]
            \item $\ph_{g} \circ \ph_{g^{-1}} = \underbrace{\ph_{e}}_{\id_{X}} = \ph_{g^{-1}} \circ \ph_{g}$
            $\implies \ph_{g}$ ist eine bijektive Abbildung mit Inverse $\varphi_{g^{-1}}$, d.h. $$\ph : G \to \Bij(X), g \mapsto \ph_{g}$$ist wohl-definiert.
            \item $\ph$ ist ein Gruppenhomomorphismus: folgt aus $(*)$ (Verknüpfung in $\Bij(X)$ ist die Verkettung von Abbildungen.)
          \end{enumerate}
        \item Übung.
  \end{enumerate}
\end{proof}
\begin{bem*}
\begin{enumerate}[(a)]
  \item Das Analogon von Proposition 2 gilt auch für Monoide. Die Linkewirkungen eines Monoids $M$ auf $X$ entsprechen Monoidhomomorphismen $M \to (\Abb(X,X), \id_{X}, \circ)$
\item Eine Gruppe kann auch auf ``Objekten'' mit mehr Struktur als eine Menge wirken, z.B. auf eine Gruppe!
\end{enumerate}
\end{bem*}
\begin{bsp*}
$G$ wirkt auf eine Gruppe $N$ heißt, man hat einen Gruppenhomomorphismus $G \to \Aut(N)$ (vgl. Lemma 1.56)
\end{bsp*}
\addcontentsline{toc}{subsection}{Eigenschaften von Wirkungen}
\begin{defi}[Eigenschaften von Wirkungen]
  Sei $\lambda: G \times X \to X$ eine Linkswirkung von $G$ auf $X$.
  \begin{enumerate}[(a)]
    \item Die Bahn zu $x \in X$ ist $Gx = \{gx \mid g \in G\}$. Die Länge der Bahn zu $x$ ist $\#Gx$
    \item $\lambda$ ist transitiv $\iff \forall y, z \in X \exists g \in G : gy=z \overset{\text{Übung}}\iff \forall y \in X : Gy = X \overset{\text{Übung}}\iff \exists x \in X : Gx = X$
    \item $\lambda$ ist $n$-fach transitiv ($n \in \N$), wenn für alle Paare von $n$-Tupeln $(x_{1}, ..., x_{n}), (y_{1}, ..., y_{n}) \in X^{n}$ mit $\#\{x_{1}, ..., x_{n}\} = \#\{y_{1}, ..., y_{n}\}$ gilt $\exists g \in G : gx_{i} = y_{i}, \forall i$.
    \item Die Wirkung heißt treu, wenn der induzierte Gruppenhomomorphismus $\ph: G \to \Bij(X)$ (aus Proposition 2) injektiv ist $$\overset{\text{Übung}} \iff \forall g \in G \setminus \{e\} : \exists x \in X : \underbrace{gX \ne X}_{\ph_{g}(x) \ne \id_{X}(x)}$$
  \end{enumerate}
\end{defi}
\begin{bsp}\item
\begin{enumerate}
  \item Ist $V$ ein $K$-Vektoraum, so wirkt das Monoid $(K, 1, \cdot)$ auf $V$ durch Skalarmultiplikation $(\lambda, v) \mapsto \lambda v$
  \item Die folgenden 3 Beispiele sind Linkswirkungen von $\GLn(K)$:
    \begin{enumerate}[(i)]
      \item $\GLn(K) \times K^{n} \to K^{n}, (g,v) \mapsto gv$.
            (Übung: Es gibt die Bahnen $\{0\}, K^{n} \setminus \{0\}$)
      \item Sei $\mathcal B = \{\text{geordnete Basen von } K^{n}\}$ und
            $$\GLn(K) \times  \mathcal B\to \mathcal B, (g, (b_{1}, ..., b_{n})) \mapsto (gb_{1}, ..., gb_{n})$$
            die Wirkung ist treu und transitiv.
      \item $\GLn(K) \times \End_{K}(K^{n}) \to \End_{K}(K^{n}), (A,B) \mapsto ABA^{-1}$ die Wirkung ist nicht treu $Z(\GLn(K))$ wirkt trivial. (Übung: Bahnen stehen in Bijektion zu den Frobeniusnormalformen von Matrizen.)
    \end{enumerate}
  \item $S_{n} \times \{1, ..., n\} \to \{1, ..., n\}, (\sigma, i) \mapsto \sigma(i)$ Wirkung ist treu und $n$-fach transitiv.
  \item Abstrakte Beispiele: Sei $H \le G$ eine Untergruppe.
        \begin{enumerate}[(i)]
          \item $\lambda: H \times G \to G, (h,g) \mapsto hg$. Die Bahnen sind die Mengen $Hg$, also die Rechtsnebenklassen zu $H$ (treu?) Menge der Rechtsnebenklassen $$\mfaktor HG := \{Hg \mid g \in G\}$$
          \item $\rho : G \times H \to G, (g,h) \mapsto gh$ Bahnen = Linksnebenklassen zu $H$ und $$\faktor GH = \{gH \mid g \in G\}$$
          \item $c_{\cdot}: G \times G \to G, (g, g') \mapsto gg'g^{-1}$ ist eine Linkswirkung, denn der nach Proposition 2 zugehörige Gruppenhomomorphismus ist $c: G \to \Aut(G), g \mapsto c_{g}$.
          \item $G \times \faktor GH \to \faktor GH, (g, g'H) \mapsto gg'H$ Die Klassen $gH$ heißen Linksnebenklassen wegen der Links-$G$-Wirkung auf ihnen.
        \end{enumerate}
\end{enumerate}
\end{bsp}

\begin{prop}
  Sei $X$ eine Links-$G$-Menge (zu der Wirkung $\lambda: G \times X \to X, (g,x), \mapsto gx$) definiere Relation $\sim$ auf $X$ durch
  $$x \sim y \iff \exists g \in G : gx = y$$
  dann gelten:
  \begin{enumerate}[(a)]
    \item $\sim$ ist eine Äquivalenzrelation.
    \item Die Äquivalenzklasse zu $x \in X$ bezüglich $\sim$ ist die Bahn $Gx$. Insbesondere ist $X$ die disjunkte Vereinigung seiner Bahnen. (Ist $(x_{i})_{i \in I}$ ein Repräsentantensystem der $G$-Bahnen, so gilt also $\#X = \sum_{i \in I}\#Gx$)
  \end{enumerate}
\end{prop}
\begin{proof}[Beweis]
  \begin{enumerate}[(a)]
    \item $\sim$ ist eine Äquivalenzrelation: Prüfe
          \begin{itemize}
          \item $\sim$ reflexiv: $ex = x \implies x \sim x.$
          \item $\sim$ symmetrisch: Gelte $x \sim y$, d.h. $\exists g \in G : gx = y$, dann gilt $x = ex = g^{-1}(gx) = g^{-1}y \implies y \sim x$.
            \item $\sim$ transitiv: Gelte $x \sim y$ und $y \sim z$, d.h. $\exists g, h' \in G : gx = y, g'y = z$
                  $$\implies (g'g)x = g'(gx) = g'y = z \implies x \sim z$$
          \end{itemize}
    \item Sei $x \in X$, dann ist $$\{y \in X \mid x \sim y\} = \{y \in X \mid \exists g \in G : y = gx\} = \{gx \mid g \in G\} = Gx.$$
  \end{enumerate}
\end{proof}
\addcontentsline{toc}{subsection}{Satz von Cayley}
\begin{satz}[\textbf{Satz von Cayley}]
\label{satz:Satz von Cayley}
Jede Gruppe $G$ (jedes Monoid $M$) ist isomorph zu einer Untergruppe (einem Untermonoid) von $(\Bij(G), \id_{G}, \circ)$ (bzw. $(\Abb(G,G), \id_{G}, \circ)$).
\end{satz}
\begin{proof}[Beweis](Für Gruppen, Rest ist eine Übung) Definiere die Wirkung $\lambda G \times G \to G, (g,h) \mapsto gh$, dann erhalten wir den induzierten Gruppenhomomorphismus $\varphi: G \to \Bij(G)$, wir zeigen $\varphi$ ist injektiv: Sei $g \in G \setminus \{e\}$, dann gilt $ge = g \ne e \implies$ Wirkung treu, also $\varphi$ ist ein Gruppenmonomorphismus. D.h. $G$ ``ist'' Untergruppe von $\Bij(G)$.
\end{proof}

\addcontentsline{toc}{subsection}{Stabilisator}
\begin{defi}[\textbf{Stabilisator}]
  Sei $X$ eine Links-$G$-Menge und $x \in X$, dann heißt
  \[G_{x} := \stb_{G}(x) := \{g \in G \mid gx = x\}\]
  Stabilisator von $x$ (unter $G$).
  Warnung: $G_{x} \ne G\cdot x$.
\end{defi}
\begin{bsp*}
  $\stb_{\Sn}(\{n\}) = \{\sigma \in \Sn \mid \sigma(n) = n\} \cong S_{n-1}$ mit der üblichen $\Sn$-Wirkung auf $\{1, ..., n\}$.
\end{bsp*}
\begin{ubng*}
  $G$-Wirkung auf einer Menge $X$ ist treu
  \[\iff \bigcap_{x \in X} \stb_{G}(x) = \{e\}\]
\end{ubng*}
\begin{prop}
  Sei $X$ eine links-$G$-Menge, $x \in X, g \in G$, dann gilt
  \begin{enumerate}[(a)]
    \item $\stb_{G}(x) \le G$ ist eine Untergruppe.
    \item $\stb_{G}(gx) = g\stb_{G}(x)g^{-1}$
  \end{enumerate}
\end{prop}
\begin{proof}[Beweis] \item
  \begin{enumerate}[(a)]
    \item $e \in \stb_{G}(x)$, denn $ex = x$. Seien $\ubr{g_{1}, g_{2} \in \stb_{G}(x)}_{\substack{\text{bedeutet } g_{1}x = x, g_{2} x= x}}$, zu zeigen ist $g_{1}^{-1}g_{2} \in \stb_{G}(x)$
          \[\overset{g_{1}^{-1}\cdot \underline\ }\imp x = ex = g_{1}^{-1}g_{1}x = g^{-1}x\]
          Damit gilt $(g_{1}^{-1}\cdot g_{2}^{-1})x = g_{1}^{-1}(g_{2}x) = g_{1}^{-1}x = x$
    \item Sei $h \in G$, dann:
          \[h \in \stb_{G}(gx) \iff hgx = gx \overset{g^{-1}\cdot \underline\ } \iff g^{-1}hgx = x \]
          \[ \iff g^{-1}hg \in \stb_{G}(x) \underset{\text{Konj. mit } g} \iff h \in g \stb_{G}(x)g^{-1}. \qedhere\]
  \end{enumerate}
\end{proof}

\begin{prop}[Bahngleichung]
  Sei $X$ eine links-$G$-Menge, $x \in X$, dann gilt:
  \begin{itemize}
    \item $\psi : \fak{G}{G_{x}} \to Gx, hG_{x} \mapsto hx$ ist wohl-definiert und eine Bijektion.
    \item Ist $G$ endlich, so folgt $\# G\cdot x = [G : G_{x}]$.
  \end{itemize}
\end{prop}
\begin{proof}[Beweis] \item
  \begin{itemize}
    \item $\psi$ injektiv und wohl definiert: Seien $g, h \in G$, dann
          \[hx = gx \iff g^{-1}hx = x \iff g^{-1}h \in G_{x} \le G\]
          \[\iff g^{-1}h G_{x} = G_{x} \iff hG_{x} = gG_x\]
    \item $\psi$ surjektiv nach Definition von $G\cdot x$.
    \item Aussage über Mächtigkeiten: $\psi$ bijektiv $\imp \# \fak G{G_{x}} = \#G \cdot x$.
  \end{itemize}
\end{proof}

\begin{bem*} Die Abbildung $\psi$ ist ein Homomorphismus von links-$G$-Mengen (ein Isomorphismus!), $\fak G{G_{x}}$ und $G \times x \subseteq X$ sind links-$G$-Mengen und $\psi$ erfüllt:
  \[\psi (g \cdot h G_{x}) = g\cdot  \psi (hG_{x})\]
  (beides ist $=gx \cdot x$)
\end{bem*}


\begin{defi}% definition 10
  Sei $X$ eine links-$G$-Menge,
  \begin{enumerate}[(a)]
    \item Man sagt $G$ operiert frei auf $X \iff \forall x \in X : G_{x} = \{e\}$
    \item Die Menge der Fixpunkte der $G$-Wirkung ist \[X^{G} := \{x \in X \mid G_{x} = G\}\]
  \end{enumerate}
\end{defi}

\begin{bsp*}
$\GLn(K)$ operiert frei auf der Menge der geordneten Basen von $K^{n}$.
\end{bsp*}

\begin{kor}
  Sei $X$ eine links-$G$-Menge. Sei $x_{1}, ..., x_{n}$ ein Repräsentantensystem der Bahnen der Länge $\ge 2$. Dann:
  \begin{enumerate}[(a)]
  \item \(X = X^{G} \sqcup \bigsqcup_{i \in \{1, ..., n\}} G \cdot x_{i}\)
  \item \(\# X = \# X^{G} + \sum_{i \in \{1, ..., n\}}\ubr{[G:G_{x_{i}}]}_{=\#G \cdot x}\)
  \end{enumerate}
\end{kor}
\begin{proof}[Beweis]
Aus Proposition 5 folgt (a), Lemma 9 gibt (b).
\end{proof}
\begin{anw*}
Sei $X:= G$. Sei die $G$-Wirkung durch Konjugation gegeben, d.h. \[g \ubr {\circ}_{\text{Wirk.}} h = ghg^{-1}\]
Die Bahnen unter dieser $G$-Wirkung heißen Konjugationsklassen. Die Konjugationsklasse zu $h \in G = X$ ist \[G_{h} := \{ghg^{-1} \mid g \in G\}\]
Bahnen der Länge 1 sind Fixpunkte unter Konjugation mit allen $g \in G$
\[= \{h \in G \mid \forall g \in G : \ubr{ghg^{-1} = h}_{gh = hg} \} =: Z(G) \text{ das Zentrum von }G\]
Stabilisator zu $h \in G$ (unter Konjugationswirkung)
\[ = \{g \in G \mid ghg^{-1} = h\} = C_{G}(h) \text{ Zentralisator von } h\]
\end{anw*}
Aus Korollar 11 ergibt sich nun:
\begin{satz}[Klassengleichung]
  Sei $G$ endlich. Ist $g_{1}, ..., g_{n}$ ein Repräsentantensystem der Konjugationsklassen der Länge $\ge 2$, so gilt:
  \[\# \ubr{G}_{X} = \#\ubr{Z(G)}_{X^{G}} + \sum_{i = 1}^{n}[G : \ubr{C_{G}(g_{i})}_{C_{g}}]\]
\end{satz}

\begin{defi}
Sei $p$ eine Primzahl, eine Gruppe $G$ heißt $p$-Gruppe $\iff \# = p^{m}$ füe ein $m \in \N$
\end{defi}
\begin{bsp*}
  \[\fak \Z{(p^{m})} \text{ oder } U_{3}(\Bbb F_{p}) =
    \l\{\begin{pmatrix}
      1 & a & b \\
      0 & 1 & c \\
      0 & 0 & 1
  \end{pmatrix} \ \middle | \ a, b, c \in \Bbb F_{p} \r\}\]
\end{bsp*}

\begin{kor}
  Ist $G$ eine $p$-Gruppe, so gilt $p | \#Z(G)$, (d.h. $Z(G)$ ist nicht-trivial und also eine $p$-Gruppe)
\end{kor}

\begin{proof}[Beweis]
  Seien $g_{1}, ..., g_{n}$ wie im Satz 12. Dann gilt: $C_{G}(g_{i}) < G$ ist eine echte Untergruppe. (sonst $g_{i} = Z(G)$, ist ausgeschlossen)
  \[\underset{\text{Lagrange}}\imp [G: C_{G}(g_{i})] \text{ teilt } \# G = p^{m}\]
  ist ungleich 1!
  \[ \imp p | [G: C_{G}(g_{i})], \forall i \in \{1, ..., n\}\]
  Klassengleichung modulo $p:$
  \[\ubr 0_{\# G} \cong \# Z(G) + \sum_{i=1}^n \ubr 0_{[G : C_{G}(g_{i})]} \mod p \imp p | \# Z(G). \qedhere\]
\end{proof}

\begin{ubng}[Satz von Cauchy](?) Sei $p$ eine Primzahl und $G$ endlich, dann gilt:
  \[p | \#G \imp \ex g \in G : \ord(g) = p.\]
  ($\imp \#G$ und $\#\exp(G)$ haben dieselben Primteiler)

  Idee: Verwende Induktion über $\#G$ und die Klassengleichung. In Induktionsschritt 2 Fälle:
  \begin{enumerate}
\item $\ex H < G$ echte Untergruppe mit $p | \#H$
\item $\neg \ex H < G$ echte Untergruppe mit $p | \#H$
  \end{enumerate}
  Im 2. Fall wende Klassengleichung mod $p$ an!
\end{ubng}

\section{Permutationsgruppen}
Sei $n \in \N$, $\Sn = \Bij(\{1, ..., n\})$, Notation für $\sigma \in \Sn$, d.h. $\sigma : \{1, ..., n\} \to\{1, ..., n\}$ bijektiv ist
\[\begin{pmatrix}
1 & 2 & \cdots & n \\
\sigma(1) & \sigma(2) & \cdots & \sigma(n) \\
  \end{pmatrix}\]
Dabei gilt: $(\sigma(1), ..., \sigma(n))$ ist eine Permutation von $\{1, ..., n\}$, d.h. \[\#\{\sigma(1), ..., \sigma(n)\} = n\]

\begin{kor}
  $\#S_{n} = n!$
\end{kor}
\begin{proof}[Beweis](Übung) Betrachte die möglichen ``Wertetabellen'' für Permutationen.
\end{proof}

\begin{defi}
  Für $\sigma, \tau \in \Sn$ definiere
  \begin{enumerate}[(a)]
    \item $\supp(\sigma) =$ Träger von $\sigma, \supp(\sigma) := \{i \in \{1, ..., n\} \mid \sigma(i) \ne i\}$
    \item $\sigma$ und $\tau$ sind disjunkt $\iff \supp(\sigma) \cap \supp(\tau) = \emptyset$
  \end{enumerate}
\end{defi}

\begin{bem*}
$\supp(\sigma) = \emptyset \iff 0 = \id$
\end{bem*}

\begin{lemm}[Andere Interpretation des Trägers]
  Sei $\sigma \in \Sn$, dann gilt für die Wirkung von $\< \sigma \> : \supp(\sigma) =$ Vereinigung der Bahnen von $\< \sigma \>$ auf $\{1, ..., n\}$ der Länge $\ge 2$.
\end{lemm}
\begin{proof}[Beweis] \item
\begin{itemize}
\item ``$\subseteq$'': Sei $i \in \supp(\s) \imp \s(i) \ne i \imp \{i, \sigma(i), \sigma^{2}(i), ..., \sigma^{m}(i), ...\}$ ist Bahn von $\<\s\> = \{\s^{j} \mid j \in \Nn\} = \{\id, \s, ..., \s^{r-1}\}$ der Länge $\ge 2$. für $r = \ord(\s)$.
\item ``$\supseteq$'': Sei $i \notin \supp(\s) \imp \s(i) = i \imp \s^{j}(i) = i, \forall j \in \N \imp$ Bahn von $i$ unter $\<\s\>$ ist 1-elementig.
\end{itemize}
\end{proof}

\begin{kor}
  Für $\s \in \Sn$ gelten:
  \begin{enumerate}[(a)]
    \item $i \in \supp (\s) \iff \s(i) \in \supp(\s)$
    \item  Auf jeder $\<\s\>$-Bahn (durch $i \in \{1, ..., n\}$) wirkt $\s$ als ``zyklische Permutation'', d.h.
          \[\begin{tikzcd}
i_n:=i \arrow[r, maps to] & i_2=\sigma(i) \arrow[r, maps to] & i_3 = \sigma^2(i) \arrow[r, maps to] & \cdots \arrow[r, maps to] & i_r = \sigma^{r-1}(i) \arrow[llll, "\sigma"', "(\text{mit } \# \{1\cdots  n\} = r)", maps to, bend left]
\end{tikzcd}\]
  \end{enumerate}
\end{kor}
\begin{proof}[Beweis]
\begin{enumerate}[(a)]
  \item \[i \in \supp(\s) \imp \s(i) \ne i \underset{\s \text{ anwenden}} \imp \s(\s(i)) \ne \s(i) \imp \s(i) \in \supp(\s)\]
        Falls $\s(i) \in \supp(\s)$, so gilt $\s(\s(i)) \ne \s(i)$
        \(\underset{\s^{-1} \text{ anwenden}} \imp \s(i) \ne i\)
  \item Sei $r$ die Länge der Bahn durch $i$ unter $\<\s\>$. Dann sind $i_{j+1}:= \s^{j}(i), j = 0, ..., r-1$ \emph{paarweise verschieden}. Sonst $\ex 0 \le j_1 < j_{2} \le r-1$ mit $\s^{j_{1}}(i) = \s^{j_{2}}(i)$
        \[\underset{\s^{-1} \text{ anwenden}} \imp i = \s^{j_{2} - j_{1}}(i) \quad (*)\]
        \(\imp\) Bahn durch $i$ hat höchstens $j_{2} - j_{1} < r$ Elemente, die Bahn ist wegen $(*)$ \[= \{i, \s(i), ..., \s^{j_{2}-j_{1}}(i)\}\]
        Und nun: Wiederholtes Anwenden von $\s$ gibt den Zykel
        \[\begin{tikzcd}
i_1 \arrow[r, maps to] & i_2 \arrow[r, maps to] & \cdots \arrow[r, maps to] & i_r \arrow[lll, bend left]
\end{tikzcd} \qedhere \]
\end{enumerate}
\end{proof}

\begin{lemm}
  Sind $\s, \tau \in \Sn$ disjunkt, so gilt $\s \tau = \tau \s$.
\end{lemm}

\begin{proof}[Beweis]
  Zeige $\sigma \circ \tau = \tau \circ \s$ als Abbildungen $\{1, ..., n\} \to \{1, ..., n\}$, sei $i \in \{1, ..., n\}$
  \begin{itemize}
    \item Fall 1: $i \in \supp(\s) \imp \s(i) \in \supp(\s) \imp i, \s(i) \notin \supp(\tau)$.
          Also $\tau(i) = i, \tau(\sigma(i)) = \sigma(i)$
    \item Fall 2: $i \in \supp(\tau)$ analog zu Fall 1.
    \item Fall 3: $i \notin \supp(\s) \cup \supp(\tau) \imp \s(i) = i = \tau(i)$.
  \end{itemize}
Also $\s(\tau(i)) = \s(i) = i = \tau(i) = \tau(\sigma(i)).$
\end{proof}
(Folge: $\s, \tau$ disjunkt $\imp \ord(\s \tau) = \kgv(\ord(\s), \ord(\tau))$)

\begin{defi}
  Seien $i_{1}, ..., i_{r} \in \{1, ..., n\}$ paarweise verschieden. Der $r$-Zykel \[(i_{1} \ i_{2}\ \cdots \ i_{r})(j) =
  \begin{cases}
    j & j \notin \{i_{1}, ..., i_{r}\} \\
    i_{s+1} & j = i_{s} \ (s \in \{1, ..., n\}) \\
    i_{1} & j = i_{r}
  \end{cases}\]
$2$-Zykel heißen Transposition.
Konvention: $( \cdot ):= \id_{\{1, ..., n\}}$ (leerer Zykel).
Beachte:
\begin{enumerate}[(i)]
  \item $(i) = (\cdot)$ für $i \in \{1, ..., n\}$
  \item $\supp(i_{1} \ i_{2} \ \cdots \ i_{r}) =
        \begin{cases}
          \{i_{1}, ..., i_{r}\} & r \ge 2 \\
          \emptyset & r =1
        \end{cases}$
  \item $(i_{1} \ i_{2}\ \cdots \ i_{r}) = (i_{r} \ i_{1} \ i_{2} \ \cdots i_{r-1})$ (Notation ist nicht eindeutig, können Einträge zyklisch weiterschieben.)
        z.B.
        \[(1 \ 4\ 7) = (7\ 1\ 4) = (4\ 7\ 1)
= \begin{tikzcd}
             & 1 \arrow[rd] &              \\
7 \arrow[ru] &              & 4 \arrow[ll]
\end{tikzcd}
        \]
  \item $\ord(i_{1}\ \cdots \ i_{r}) = r$, z.B. $\ord(1 \ 2) = 2$
\end{enumerate}
\end{defi}

\begin{satz}[Zykeldarstellung]
  Sei $\s \in \Sn$, seien $I_{1}, ..., I_{t} \subseteq \{1, ..., n\}$. Die verschiedenen Bahnen von $\< \s \>$ der Länge $\ge 2$, dann:
  \begin{enumerate}[(a)]
    \item $\ex !$ Zykel $\s_{j}$ der Länge $\# I_{j}$ mit $\supp(\s_{j}) =I_{j}$, so dass $\s_{j}|_{I_{j}} = \s|_{I_{j}}$
    \item $\s = \s_{1} \cdot ... \cdot \s_{t}$ und die $\s_{i}$ kommutieren paarweise.
    \item Die Darstellung in (b) ist bis auf Permutation der Faktoren eindeutig.
    \item Es gilt mit der Notation aus (b):
          \[\ord(\s) = \kgv(\#I_{1}, ..., \# I_{t})\]
  \end{enumerate}

\end{satz}


\end{document}
