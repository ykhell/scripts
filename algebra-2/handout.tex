\documentclass{article}
\usepackage{template}
\usepackage[english]{babel}
\usepackage[utf8]{inputenc}
\usepackage{geometry}
\usepackage[T1]{fontenc}
\usepackage{multicol}
\usepackage{fancyhdr}

%%%%%%%%%%%%%%%%%%%%%%%%%%%%%%%%%%%%%%%%%%%%%%%%%%%%%%%

\begin{document}
\pagestyle{fancy}
\renewcommand{\footrulewidth}{0.4pt}
\fancyhead{}
\fancyhead[RO,RE]{Representation theory of finite groups}
\fancyhead[LO,L]{Yousef Khell}
\fancyfoot{}
\fancyfoot[LO]{Handout: Examples of induced representations} %This is the place for footnote
    \begin{center}
    	\Huge{\textbf{Talk 10: Examples of induced representations}}%Title spot
      \end{center}

      \begin{multicols*}{2}
        \begin{rem*}
            In this talk all groups are finite and all representations are finite dimensional over $\mathbb C$.
        \end{rem*}
        \section{Normal subgroups}
        \begin{defi}[Normal subgroup]
          % A subgroup $N \le G$ is called \emph{normal}, if its invariant under conjugation from $G$, i.e. if $gNg^{-1} \subseteq N,\forall g \in G$, and we write $N \nt G$.
          Recall that $N \nt G :\iff gNg^{-1} = N, \forall g \in G$.
        \end{defi}
        \begin{defi}[Isotypical representation]
          A representation is called \emph{isotypical} if its a direct sum of isomorphic irred. representations.
        \end{defi}
        \begin{prop}
          Let $A \nt G, \rho: G \to \GL(V)$ be an irreducible representation, then either
          \item (a) $\ex H \lneq G$ containing $A$ and an irred. $\s : H \to \GL(W)$ so that $\rho = \ind_{H}^{G}(\s)$, or
          \item (b) the restriction $\rho|_{A}$ is isotypic.
        \end{prop}

\begin{rem}
If $A$ is abelian, then (b) $\iff \rho(a)$ is a homothety $\forall a \in A$.
\end{rem}
        \begin{cor}
If $A \nt G$ is abelian, then $\forall \rho$ irred. representation of $G$ we have $\deg \rho \mid [G:N]$.
\end{cor}

\begin{rem}
  If $A \le G$ is an abelian subgroup (not necessarily normal), this doesn't hold in general, but we still have the bound $\deg \rho \le [G: A]$ like we have seen in corollary 1.3. from talk 4.
\end{rem}

\section{Semidirect products by an abelian group}
\begin{const}
    Let $G = A \rtimes H$, where $A \nt G$ is an abelian normal subgroup.
  \begin{enumerate}[(1)]
    \item Let $X:= \Hom(A,\mathbb C\en)$ be the set of irred. characters of $A$, and let $G \curvearrowright X$ by $s\cd \chi(a) = \chi(s^{-1}as)$.
    \item Let $(\chi_{i})_{i \in X/H}$ be representatives for the orbits of the action $H \curvearrowright X$, and write $H_{i} := \stb_{H}(\chi_{i})$ and $G_{i} := A \rtimes H_{i}$.
    \item Extend $\chi_{i}$ to $G_{i}$ by $\tilde \chi_{i}(a\cd h) = \chi_{i}(a)$.
    \item Let $\rho: H_{i} \to \GL(W)$ be irreducible and extend it to an irred. representation of $G_{i}$ by $\tilde \rho(a \cd h) = \rho(h)$.
    \item The tensor $\tilde \chi_{i} \otm \tilde \rho$ will be also irreducible.
    \item Finally define $\theta_{i, \rho} = \ind_{G_{i}}^{G}(\tilde \chi_{i} \otm \tilde \rho)$.
  \end{enumerate}
\end{const}
\begin{prop}
\begin{enumerate}[(a)]
  \item $\theta_{i, \rho}$ is irred.
  \item $\theta_{i,\rho} \cong \theta_{i',\rho'} \imp i = i'$ and $\rho \cong \rho'$.
  \item These are all the irred. representations of $G$.
\end{enumerate}
\end{prop}
\begin{exmp}
Application to $D_{n}, A_{4}$ and $S_{4}$.
\end{exmp}

\section{Review on classes of groups}
\begin{defi}
  A group $G$ is called
  \begin{enumerate}[(a)]
    \item \emph{solvable}, if $G$ has a normal series
          \[\{e\} = G_{0} \nt G_{1} \nt \cdots \nt G_{n} = G\]
          where all factors $\fak {G_{i}}{G_{i-1}}$ are abelian.
    \item \emph{supersolvable}, if $G$ is solvable, all $G_{i} \nt G$ are normal in $G$ and all factors $\fak {G_{i}}{G_{i-1}}$ are cyclic.
    \item \emph{nilpotent}, if $G$ has a normal series with $\fak{G_{i+1}}{G_{i}} \le Z(\fak{G}{G_{i}})$
    \item \emph{a $p$-group}, if $|G| = p^{k}$ for some prime.
  \end{enumerate}
\end{defi}
\begin{rem}
(a)$\impliedby$(b)$\impliedby$(c)$\impliedby$(d)
\end{rem}
\begin{defi}
A $p$-subgroup $H \le G$ is called a \emph{Sylow $p$-subgroup} of $G$ if it's maximal.
\end{defi}

\begin{thm}[Sylow]
  Let $G$ be a group, then for each prime factor of $|G|$
  \begin{enumerate}[(a)]
    \item Sylow $p$-subgroups exist, and their number is congruent to $1 \mod p$.
    \item They are conjugates.
    \item Each $p$-subgroup is contained in a Sylow $p$-subgroup.
  \end{enumerate}

\end{thm}



\section{Representations of supersolvable groups}
\begin{defi}[Monomial]
  A group $G$ is \emph{monomial}, if $\forall$ irred. $\rho$ is induced by a degree 1 rep. of a subgroup.
\end{defi}

\begin{lemm}
Let $G$ be supersolvable and nonabelian. Then $\ex A \nt G$ abelian not contained in $Z(G)$.
\end{lemm}

\begin{thm}
Every supersolvable group is monomial.
\end{thm}
\begin{cor}If $G = A \rtimes H$, where $A \nt G$ is abelian and $H$ is supersolvable, then $G$ is monomial.

\end{cor}
% \section{Notation/Definitions}
% \begin{nota*}
% \begin{enumerate}[(a)]
%   \item $G \curvearrowright X : \iff G$ acts on $X$
%   \item $N \nt G :\iff N$ is normal in $G$
%   \item
% \end{enumerate}

% \end{nota*}
\begin{thebibliography}{widest-label} % References are listed here
	\bibitem[Ser77]{Ser77}%You can add more refernces by \bibitem command
	Jean-Pierre Serre:
	\emph{Linear Representations of Finite Groups}, Springer-Verlag, New York, 1977.

\end{thebibliography}
\end{multicols*}
\end{document}
