\documentclass[a4paper]{report}
\usepackage{../template}
\begin{document}

% Lecture 1
\section{Modules}
Let $(R, 0, 1, +, \cd)$ or simply $R$ be a ring.
\begin{defi}
\begin{enumerate}[(a)]
        \item A left $R$-module $(M, 0, +, \cd)$ or simply $M$ is an abelian group $(M, 0, +)$, together with an operation $\cd: R \tm M \to M, (r,m) \mps r \cd m = rm$, such that for all $a, b \in R, m, n \in M$
        \begin{enumerate}[(M1)]
          \item $a(m+n) = am+an$ and $(a+b)m = am + bm$
          \item $a(b \cd m) = (ab) \cd m$
                \item $1 \cd m = m$
        \end{enumerate}
  \item Let $M, N$ be left $R$-modules. A map $\ph : M \to N$ is called $R$-linear or a left $R$-module homomorphism $:\iff \ph: (M, 0, +) \to (N,0,+)$ is a group homomorphism, and $\forall a \in R, m \in M : \ph(am) = a\ph(m)$. Define $\Hom_{R}(M,N) = \{\ph: M \to N \mid \ph$ is $R$-linear$\}$.
\end{enumerate}
\end{defi}

\begin{facts}[Excers.]$\forall x \in M, a \in R: 0_{R} \cd x = 0_{M}, a \cd 0_{M} = 0_{M}, (-1)\cd x = -x$
\end{facts}
\begin{rem}[Excers.]
\begin{enumerate}[(a)]
  \item $\Hom_{R}(M,N)$ is an abelian group with $0 =$ the map $M \to \{0_{N}\}$ and $\ph + \psi: M \to N, m \mps \ph(m)+\psi(m)$.
  \item If $R$ is commutative, then $\Hom_{R}(M,N)$ is an $R$-module via \[r \cd \ph : M \to N, m \mps r \cd \ph(m)\]
  \item If an abelian group $(M,0,+)$ carries an operation $\cd: M \tm R \to M, (m,r) \mps m \cd r$ such that:
        \begin{enumerate}[(M1')]
          \item $(m+n) \cd a = m \cd a + n \cd a, m \cd (a+b) = ma + mb$
          \item $(m \cd a)\cd b = m \cd (ab)$
                \item $m \cd 1 = m$
        \end{enumerate}
then $(M,0,+,\cd)$ is called a right $R$-module. Analogously we can define right $R$-module homomorphisms.
\end{enumerate}
\end{rem}
\begin{conv} We shall use the term $R$-module for left $R$-module, since we will mainly work with these. In fact right $R$-modules are left $R^{\op}$-modules.
\end{conv}
\begin{defi}
The opposite ring (Gegenring) of $(R,0,1,+,\cd)$ is $R\op = (R,0,1,+,\cd\op)$ with $a \cd\op b = b \cd a$
\end{defi}
\begin{facts}[Excersize]
  \begin{enumerate}[(a)]
    \item $R\op$ is a ring
    \item $\id_{R}: R \to R$ is a ring homomorphism $\iff R$ is commutative.
    \item $\id_{R}: R \to (R\op)\op$ is an isomorphism.
    %\item Let $(M, 0, +)$ be an abelian group, with an operation $\cd: R \tm M \to M, (r,m) \mps m \cd\op r := r \cd m$, then $(M, 0, +, \cd)$ is a left $R$-module $\iff (M, 0, +, \cd\op)$ is a right $R\op$-module.
  \end{enumerate}
In particular: If $R$ is commutative, then left $R$-modules are right $R$-modules.
\end{facts}
\begin{rem}[Excersize]
  Let $(M, 0, +)$ be an abelian group.
  \begin{enumerate}[(a)]
    \item The abelian group $\End_{\Z}(M) = \Hom_{\Z}(M,M)$ is a ring with composition as multiplication.
    \item There is a bijection $\{$operations $*: R \tm M \to M \mid (M,0,+,*)$ is an $R$-module$\} \leftrightarrow \{$ring homomorphisms $\ph: R \to \End_{\Z}(M)\}$ via \[* \mps \ph_{*}: R \to \End_{\Z}(M), r \mps (\ph_{*}(r): m \mps r \cd m)\]figure out an inverse.
    \item If $M$ is an $R$-module, then $\End_{R}(M) \subseteq \End_{\Z}(M)$ is a subring
    \item The map $R\op \to \End_{R}(R), r \mps \rho_{r}: a \mps a \cd r$ is a ring isomorphism. The inverse is $\End_{R}(R) \to R\op, \ph \mps \ph(1)$
  \end{enumerate}
\end{rem}

\begin{exmp}
\begin{enumerate}[(a)]
        \item Let $K$ be a field, $K$-modules are $K$-vector spaces and vice versa.
  \item If $(M, 0, +)$ is an abelian group, it is in a unique way a $\Z$-module.
  \item Let $K$ be a field, $R = M_{n \tm n}(K), n > 1, V_{n}(K) =$ column $Z_{n}(K)$ row vectors of length $n$ over $K$, then:
        \begin{itemize}
          \item $V_{n}(K)$ is a left $R$-module.
                \item $Z_{n}(K)$ is a right $R$-module.
        \end{itemize}
  \item $R$ is a left $R$-module and right $R$ module with multiplication.
  \item If $M_{1}$ and $M_{2}$ are $R$-modules, we can define on $M_{1} \tm M_{2}$ a $R$-module structure via
        \[r \cd (m_{1}, m_{2}) := (rm_{1}, rm_{2})\] (group structure from Algebra 1)
        \item $\Hom_{R}(R, M) \to M, \ph: \ph(1)$ is an isomorphism of abelian groups, and if $R$ is commutative, then also an isomorphism of $R$-modules.
\end{enumerate}
\end{exmp}

\begin{defi}
An $R$-linear map $\ph: M \to M'$ is called a monomorphism/epimorphism/isomorphism $\iff \ph$ is injective/surjective/bijective respectively. We say $R$-modules $M, M'$ are isomorphic if there exists an isomorphism $M \to M'$.
\end{defi}
\begin{rem*}
$\ph$ is an $R$-linear isomorphism $\iff \ph^{-1}$ is an $R$-linear isomorphism.
\end{rem*}

\begin{defi}
\begin{enumerate}[(a)]
  \item Let $M$ be an $R$-module. A subset $N \subseteq M$ is an $R$-submodule if it is a subgroup and $\forall a \in R, n \in N : a \cd n \in N$ (i.e. $R \cd N \subseteq N$)
  \item An $R$-submodule $I \subseteq R$ is called a left ideal.
        \item $I \subseteq R$ is called a two sided ideal iff it is a left ideal and $I \cd R \subseteq I$
\end{enumerate}
\end{defi}

\begin{exmp}
\begin{enumerate}[(a)]
  \item If $N' \subseteq N$ and $M' \subseteq M$ are $R$-submodules of $R$-modules $M$ and $N$ and if $\ph: M \to N$ is an $R$-linear map, then: \[\ph(M') \subseteq N \text{ and } \ph^{-1}(N') \subseteq M\]
        are $R$-submodules. In particular $\ker(\ph) \le M$ and $im(\ph) \le N$ are submodules.
  \item If $(M_{i})_{i \in I}$ is a family of submodules of $M$, then $\bigcap_{i \in I}M_{i} \subseteq M$ is the largest submodule of $M$ contained in all $M_{i}$, and \[\sum_{i \in I} M_{i} = \{\sum_{i \in I}m_{i} \mid m_{i \in M)i}, \#\{i \mid m_{i \ne 0}\} < \infty\}\]
        is the smallest submodule of $M$ containing all $M_{i}$.
        \item 2-sided ideals of $M_{n \tm n}(R)$ are of the form $M_{n \tm n}(I)$ for $I \subseteq R$ a 2-sided ideal.
\end{enumerate}
\end{exmp}

\section{Quotient Modules}
\begin{defi}
  Let $N \subseteq N$ be a submodule. From linear algebra $(\fak MN, \bar 0, \bar +)$ is an abelian group. ($\bar m = m + N$ are the equivalence classes and $\bar m + \bar m' = \bar{m+m'}$). This is an $R$-module (exercise) via \[\bar \cd : R \tm \fak MN \to \fak MN: (r, m+N) \mps rm + N\]
  We call $\fak MN$ (with $\bar 0, \bar +, \bar \cd$) the quotient module of $M$ by $N$, and we write \[\pi_{N \subseteq M}: M \twoheadrightarrow \fak MN, m \mps m+N\]
\end{defi}
\begin{defi}
  If $I \subseteq R$ is a 2-sided ideal of $R$, then
  \begin{enumerate}[(a)]
    \item $I \cd M := \{\sum_{i \in I} a_{i} \cd m_{i} \mid I$ finite, $a_{i \in I, m_{i} \in M}\}$ is an $R$-submodule of $M$ ($M$ an $R$-module)
          \item $(\fak RI, \bar 0, \bar 1, \bar +, \bar \cd)$ is a ring, and $\fak M{I \cd M}$ is an $\fak RI$-module.
  \end{enumerate}
\end{defi}
The following 3 results are proved as for groups:
\begin{thm}[Homomorphism theorem]
  Let $\ph: M \to M'$ be an $R$-linear map, then
  \begin{enumerate}[(a)]
    \item $\forall$ submodules $N \subseteq \ker(\ph): \ex ! R$-linear map $\bar \ph : \fak MN \to M', m+N \mps \ph(m)$ such that $\ph = \bar \ph \circ \pi_{N \subseteq M}$
          \item For $N = \ker(\ph)$, the map $\bar \ph : \fak M{\ker(\ph)} \to im(\ph)$ is an $R$-module isomorphism.
  \end{enumerate}
\end{thm}
\begin{thm}(First isomorphism theorem)
  Let $M$ be an $R$-module and $N_{1}, N_{2} \le M$ be $R$-submodules. Then the map \[\fak {N_{1}}{N_{1} \cap N_{2}} \to \fak {N_{1} + N_{2}}{N_{2}}, n_{1} + N_{1} \cap N_{2} \mps n_{1} + N_{2}\]
  is a well-defined $R$-linear isomorphism.
  \[\begin{tikzcd}
                        & N_1+N_2 \arrow[ld, no head]      &                         \\
N_1 \arrow[rd, no head] &                                  & N_2 \arrow[lu, no head] \\
                        & N_1 \cap N_2 \arrow[ru, no head] &
\end{tikzcd}\]
\end{thm}
\begin{thm}[Second isomorphism theorem]
  Let $M$ be an $R$-module and $N \le M$ an $R$-submodule. Then
  \begin{enumerate}[(a)]
    \item The following maps are bijective and mutually inverse to each other:
          \begin{align*}
            \{N' \subseteq M \text{ submodule} \mid N \subseteq N'\} &{\overunderset \ph \psi \rightleftarrows} \{\bar N \subseteq \fak MN \text{ submodule}\} \\
            \ph: N {\mps} \fak {N'}N \quad & \quad\ \pi_{N \subseteq M}^{-1}(\bar N) {\mpsfrom} \bar N : \psi
          \end{align*}
    \item For $N' \subseteq M$ a submodule with $N \subseteq N'$ we have the $R$-linear isomorphism:
          \[\fak{(M/N)}{(N'/N)} \to \fak M{N'}, \bar m + \fak {N'}M \mps m+N'\]
  \end{enumerate}
\end{thm}

\section{Direct sums and products}
Let $(M_{i})_{i \in I}$ be a family of $R$-modules.
\begin{defi}
\begin{enumerate}[(a)]
  \item $\prod_{i \in I}M_{i} = \{(m_{i})_{i \in I} \mid m_{i} \in M_{i}, \forall i \in I\}$ is an $R$-module with component-wise operations:
        \[(m_{i})_{i \in I} + (n_{i})_{i \in I} =(m_{i}+n_{i})_{i \in I}\]
        \[r \cd (m_{i})_{i \in I} = (r \cd m_{i})_{i \in I}, \quad r \in R\]
        is called the (direct) product of $(M_{i})_{i \in I}$. One has the projection maps ($R$-module epimorphisms):\[\pi_{i_{0}}: \prod_{i \in I} M_{i} \to M_{i_{0}}, (m_{i}) \mps m_{i_{0}}\]
        \item $\bigoplus_{i \in I}M_{i} = \{(m_{i})_{i \in I} \in \prod_{i \in I} M_{i} \mid \{i \mid m_{i} \ne 0\} < \infty \}$ is an $R$-submodule of $\prod_{i \in I} M_{i}$. It is called the direct sum of $(M_{i})_{i \in I}$. One has $R$-module monomorphisms \[\iota_{i_{0}} : M_{i_{0}} \to \bigoplus_{i \in I} M_{i}, m_{i_{0}} \mps (\iota_{i_{0}}(m_{i_{0}}))\]
where the $i$-th component of $\iota_{i_{0}}(m_{i_{0}})$ is given by $\begin{cases} m_{i_{0}}, & i = i_{0}, \\ 0, & \text{otherwise}
\end{cases}$
\end{enumerate}
\end{defi}
\begin{thm}[Universal property of the direct product/sum]
\begin{enumerate}[(a)]
        \item $\forall R$-modules $M$, the map
        \[\Hom_{R}(M, \prod_{i \in I}M_{i}) \xrightarrow{\cong} \prod_{i \in I} \Hom_{R}(M, M_{i}), \ph \mps (\pi_{i} \circ \ph)_{i \in I}\]is well defined, bijective and a group isomorphism.
        \item $\forall R$-modules $M$, the map
        \[\Hom_{R}(\bigoplus_{i \in I}M_{i}, M) \xrightarrow{\cong} \prod_{i \in I} \Hom_{R}(M_{i}, M), \psi \mps (\psi \cd \iota_{i})_{i \in I}\]is well defined, bijective and a group isomorphism.
\end{enumerate}
\begin{proof}
  \begin{enumerate}[(a)]
    \item The inverse map is given by sending \[\underline \ph:= (\ph_{i}: M \to M_{i})_{i \in I} \in \prod_{i \in I}\Hom_{R}(M, M_{i})\]
          to \[\pi_{\underline \ph} : M \to \prod_{i \in I}M_{i}, m \mps (\ph_{i}(m))_{i \in I}\]
          now check: $\underline \ph \mps \pi_{\underline \ph}$ is inverse to the map in (a).
    \item The map is given by sending $\bar \ph = (\ph_{i} : M _{i} \to M)_{i \in I}$ to
          \[\coprod_{\bar \ph} : \bigoplus_{i \in I}: M_{i} \to M, (m_{i})_{i \in I} \mps \sum_{i \in I} \ph_{i}(m_{i})\]
          %LATER remark about finiteness
  \end{enumerate}
\end{proof}
\end{thm}

\begin{cor}[Important special case]
  Let $I$ be finite, then:
  \begin{enumerate}[(a)]
    \item $M := \prod_{i \in I} M_{i} \overset != \bigoplus_{i \in I}M_{i}$
    \item The maps $M_{i} \overunderset {\iota_{i}}{\pi_{i}}\rightleftarrows M$ satisfy \[\pi_{i} \circ \iota_{j} = \begin{cases} \id_{M_{i}},& i = j, \\ 0, &\text{otherwise}\end{cases}\quad \text{and}\quad \sum_{i \in I}\iota_{i} \circ \pi_{i} = \id_{M}\]
          \item If $M'$ is a module with maps $M_{i} \overunderset {\iota_{i}'}{\pi_{i}'}\rightleftarrows M'$ such that the formulas above hold, then $M \cong M'$
  \end{enumerate}
\end{cor}



% Lecture 2
\section{Generators and bases}
From now onwards let $R$ be a unitary ring and $M, M', N$ be $R$-modules.
\begin{nota*}
  \begin{itemize}
    \item For $I$ a set we write $M^{I}:= \prod_{i \in I}M$ and $M^{(I)}:= \bigoplus_{i \in I}M$ (where $M_{i} = M, \forall i \in I$).
          \item For $r \in \N$ we will write $M^{r}:= M^{\eb r}$, so if $I$ is finite then $M^{I} = M^{\#I} = M^{(I)}$
  \end{itemize}
\end{nota*}

\begin{defi} %20
  For $\underline m = (m_{i})_{i \in I} \in M^{(I)}$ we define a map $\ph_{\underline m}: R^{(I)} \to M, (r_{i}) \mps \sum_{i \in I}r_{i} \cd m_{i}$ where $r_{i}$ is non-zero only for finitely many $i$. %TODO organize
  We can also define $\ph_{\underline m}$ via the universal property of $R^{(I)}$ using maps $R \to M, r \mps r \cd m_{i}$ at component $i \in I$. %TODO make sure statement is right, watch the video.
  \begin{enumerate}[(a)]
    \item $\underline m$ is a generating set of $M \iff \ph_{\underline m}$ is surjective.
    \item $\underline m$ is a basis of $M \iff \ph_{\underline m}$ is an isomorphism.
    \item $M$ is a free $R$-module $\iff M$ has a basis.
    \item $\underline m$ is finitely generated $\iff$ it has a finite generating set.
    \item $\underline m$ is linearly independent $\iff \ph_{\underline m}$ is injective.
  \end{enumerate}
\end{defi}

\begin{rem*}
  Let $\iota_{j}: R \to R^{(I)}$ be the inclusion of the component $j \in I$ (1.18) and set $e_{j}:= \iota_{j}(1)$. Then we call $(e_{j})_{j \in I}$ the standard basis of $R^{(I)}$.
\end{rem*}
  \begin{exmp*}
\begin{enumerate}[(a)]
  \item If $K$ is a field, then any $K$-vector space has a basis.
        \item If $R = \Z$, then $M = \fak \Z{(3)}$ is finitely generated but not free (exercise).
\end{enumerate}

  \end{exmp*}
  \begin{rem}%21
    Every $R$-module is a quotient of a free $R$-module.
\end{rem}
\begin{proof}
Let $R^{(M)}$ be the free $R$-module over the index set $M$, then \[\ph_{\underline m}: R^{(M)} \to M, (r_{m})_{m \in M} \mps \sum_{m \in M}r_{m} \cd m\] is surjective for $\underline m = (m)_{m \in M}$.
\end{proof}
\begin{thm}%22
Let $R$ be commutative, then for $n_{1}, n_{2} \in \Nn$, then we have $R^{n_{1}} \cong R^{n_{2}} \iff n_{1} = n_{2}$.
\end{thm}
\begin{proof}
\begin{itemize}%TODO look at the whole thing
  \item ``$\impliedby$'': (By induction to linear algebra.) Let $\m \subseteq R$ be a maximal ideal. (Axiom of choice) Consider for $n \in \N$ the map $\ph_{n}: R^{n} \to (\fak R\m)^{n}, (r_{1}, \ldots r_{n}) \mps (r_{i} \mod n)_{i \in \eb n}$. Then $\ph_{n}$ is surjective with kernel $\m^{n} \in R^{n} \imp \fak{R^{n}}{\m^{n}} \cong (\fak R\m)^{n}$ by the homomorphism theorem. Now suppose $\psi: R^{n_{1}} \to R^{n_{2}}$ is an isomorphism. We show $n_{1} \ge n_{2}$ (by symmetry of argument we get $n_{1} = n_{2}$).
        Consider the map
        \[\begin{tikzcd}
R^{n_1} \arrow[r, "\cong"] \arrow[rr, "\rho"', bend right] & R^{n_2} \arrow[r, two heads] & \displaystyle\fak{R^{n_2}}{\m^{n_2}}
\end{tikzcd}\]
        this map is surjective and contains $\m^{n_{1}}$ in its kernel (check this). By the homomorphism theorem we get a surjective homomorphism \[\l(\fak R\m\r)^{n_{1}} = \fak{R^{n_{1}}}{\m^{n_{1}}} \to \fak{R^{n_{2}}}{\m^{n_{2}}} = \l(\fak R\m\r)^{n_{2}}\]
        by linear algebra we conclude that $n_{1} \ge n_{2}$.
\end{itemize}
\end{proof}

\begin{defi}%23
If $M$ is free and finitely generated, then define $\mathrm{rank}(M)$ (the rank of $M$) as the unique $n \in \Nn$ such that $M \cong R^{n}$.
\end{defi}

\begin{rem*}
If $R$ is non-commutative, then the rank of the finitely generated $R$-modules is not well-defined. (Jantzen Schwermer Bsp VII.4.2: $R \cong M \cong R^{2}$ for $R = \End_{K}(K[X])$)
\end{rem*}

\section{Exact sequences}
\begin{defi}
\begin{enumerate}[(a)]
  \item A diagram of $R$-modules \[M' \xrightarrow{f} M \xrightarrow{g}M''\]
        is called exact (at $M$) $:\iff \ker(g) = \mathrm{im}(f)$
  \item An exact sequence of $R$-modules is a family $(f_{j})_{j \in J}$ of $R$-module homomorphisms $f_{j}: M_{j} \to M_{j+1}$ index of an interval $J \subseteq \Z$, such that $\forall j \in J : j+1 \in J$, the sequence
        \[M_{j} \xrightarrow{f_{j}} M_{j+1} \xrightarrow{f_{j+1}}M_{j+2}\]
        is exact (at $M_{j+1}$). Other notation: %Read what he says
        \[M_{j_{0}} \xrightarrow{f_{j_{0}}} M_{j_{0}+1} \xrightarrow{f_{j_{0}+1}} \cdots \to M_{j+2}\]
        \item An exact sequence $0 \to M' \to M \to M'' \to 0$ is called a short exact sequence (s.e.s.)
\end{enumerate}
\end{defi}

\begin{rem*}
  \begin{itemize}
  \item $0 \to M' \xrightarrow{f} M$ is exact $\overset{\text{Exercise}}\iff f$ is injective.
  \item $M \xrightarrow{g} M'' \to 0$ is exact $\overset{\text{Exercise}}\iff g$ is surjective.
  \end{itemize}
  ($0$ stands for the $0$-module $\{0\}$)
\end{rem*}
\begin{exmp}
  Let $f: M \to N$ be an $R$-module homomorphism. Then one defines
  \[\mathrm{coker}(f):= \fak N{\mathrm{im}(f)}\]
 as the cokernel of $f$ , it comes toether with an $R$-module epimorphism $\pi: N \to \mathrm{coker}(f)$. As an exercise: The sequence
  \[0 \to \ker(f) \xrightarrow{i} M \xrightarrow f N \xrightarrow \pi \mathrm{coker}(f) \to 0\]
  is exact. Subexamples:
  \begin{itemize}
    \item If $f$ is injective, then $0 \to M \xrightarrow f N \to \mathrm{coker}(f) \to 0$ is exact.
          \item If $f$ is surjective, then $0 \to \ker(f) \to M \xrightarrow f N \to 0$ is exact.
  \end{itemize}
\end{exmp}

\begin{rem*}
  For $R$-module homomorphisms $M' \xrightarrow \a M \xrightarrow \b M''$ with $\b \circ \a = 0$, the following are equivalent:
  \begin{enumerate}[(i)]
    \item $0 \to M' \xrightarrow \a M \xrightarrow \b M'' \to 0$ is a s.e.s.
    \item $\b$ is surjective and $\a: M' \to \ker(\b)$ is an isomorphism.
          \item $\a$ is injective and the homomorphism theorem induces an isomorphism $\mathrm{coker}(\a) \cong \fak M{\mathrm{im}(\a)} \to M''$
  \end{enumerate}
  ($\b \circ \a = 0 \iff \mathrm{im}(\a) \subseteq \ker(\b)$)
\end{rem*}


\begin{prop}[Exercise]
\begin{enumerate}[(a)]
  \item Let $0 \to M_{i}' \to M_{i} \to M_{i}'' \to 0$ be short exact sequences $\forall i \in I$, then we get short exact sequences
        \[0 \to \bigoplus_{i \in I}M_{i}' \to \bigoplus_{i \in I}M_{i} \to \bigoplus_{i \in I}M_{i}'' \to 0\]
        \[0 \to \prod_{i \in I}M_{i}' \to \prod_{i \in I}M_{i} \to \prod_{i \in I}M_{i}'' \to 0\]
      \item Suppose $0 \to V_{0} \xrightarrow {f_{0}} V_{1} \xrightarrow{f_{1}} \cdots \xrightarrow {f_{n-1}} V_{n} \to 0$ is an exact sequence of finite dimensional $K$-vector spaces, then: \[\sum(-1)^{i} \dim_{K}(V_{i}) = 0.\]
\end{enumerate}

\end{prop}

\begin{nota}[Commutativity of diagrams]
A diagram of $R$-modules is a directed graph, where any vertex is an $R$-module and any arrow is an $R$-linear map from the module at its source to the module at its target. We call two arrows composable if the target of the first arrow is the source of the second; then the correspoding maps can be composed. So to any chain of composable arrows, the composition of maps defines a map from the source of the first to the target of the last arrow in the chain. A diagram is \textbf{commutative} if for any two chains of arrows with the same source and target, the resulting two maps agree.
\end{nota}
\begin{exmp*}
  \begin{enumerate}[(a)]
  \item To say that the diagram
\[\begin{tikzcd}.
M_1 \arrow[r, "f"] \arrow[d, "g"'] & M_2 \arrow[d, "g'"] \\
M_3 \arrow[r, "f'"']               & M_4
\end{tikzcd}\]
commutes means that $g' \circ f = f' \circ g$.
\item $M \overunderset fg \rightleftarrows N$ commutes $\iff g = h$
   \end{enumerate}
 \end{exmp*}

\begin{thm-defi}
  For a short exact sequence of $R$-modules %TODO fix the thing like in the VL
  \[\begin{tikzcd}
0 \arrow[r] & M' \arrow[r, "f"] & M \arrow[r, "g"] \arrow[l, "s", dashed, bend left] & M'' \arrow[r] \arrow[l, "t", dashed, bend left] & 0
\end{tikzcd} \quad (*)\]
  the following are equivalent:
  \begin{enumerate}[(a)]
    \item $\ex R$-linear map $t: M'' \to M$ such that $g \circ t = \id_{M''}$
    \item $\ex$ submodule $N \subseteq M$ such that \[\psi: \im(f) \oplus N \to M, (b,n) \mps b+n\]
          is an isomorphism.
          \item $\ex R$-linear map $s: M \to M'$ such that $s \circ f = \id_{M'}$.
  \end{enumerate}
  In this case (if (a) - (c) hold), then the sequence $(*)$ is called a split exact sequence. (simply $(*)$ is split or splits), and $t$ (or $s$) is called a splitting of $g$ (or of $f$ respectively).
  \begin{proof}
    \begin{itemize}
      \item $(a){\imp}(b)$: Given $t$, define $N:= \im(t)$ and $\psi$ as above, i.e. $\psi: \im(f) \oplus N \to M, (b,n) \mps b+n$
            \begin{itemize}
              \item $\ker(\psi) = 0$: Let $(b,n) \in \ker(\psi)$, i.e. $n = t(m'')$, for some $m'' \in M''$ and $b = f(m')$ for some $m' \in M'$ and $n+b = 0$ ($\psi(b,n) = 0$).
              \item Apply $g: M \to M''$: \[\ubr{g(n+b)}_{0} = \ubr{g(t(m''))}_{g \circ t = \id_{M''}} + \ubr{g(f(m'))}_{g \circ f = 0} = m''+0\]
                    $\imp m'' = 0 \imp n = t(m'') = 0 \underset{n+b=0}\imp b=0 \imp (b,n) = (0,0)$
                    \item $\im(\psi) = M$: Let $m \in M$, define $n = t(g(m))$ and $b = m-n$. So $n \in N = \im(f)$. $b \in \im(f)$?, to show $b \in \ker(g)$. For this $g(b) = g(m-n) = g(m) - \ubr{g(t(g(m)))}_{g\circ t = \id_{M''}} = g(m) - g(m)= 0$, so $(b,n) \in \im(f) \oplus N$ and $\psi(b,n) = b+n = m$ by definition of $b$.
            \end{itemize}
      \item $(c) {\imp} (b)$ analogous. Define $N = \ker(s)$ ($M' \overunderset fs \rightleftarrows M$). We want to show $\im(f) \oplus N \to M, (b,n) \to b+n$ is an isomorphism.
            \begin{itemize}
              \item $\ker(\psi) = 0$: Check.
              \item $\im(\psi) = M$: For $m \in M$ observe that \[\ubr{f \circ s(m)}_{\in \im(f)} + \ubr{(m-f\circ s(m))}_{\in \ker(s) \text{ check. }} = m\]
            \end{itemize}

      \item $(b) \to (a)$ and $(c)$: Consider the diagram:
            \[
\begin{tikzcd}[column sep= 20, row sep = 20]
0 \arrow[r] & M' \arrow[r, "f'"] \arrow[d, "\id_{M'}"] \arrow[r, "{\small \ \ \ \ m' \mps (f(m'),0)}"'] & \im(f) \oplus N \arrow[r, "g'"] \arrow[d, "\psi"] \arrow[l, dashed, bend right] \arrow[r, "{\small (b,n) \mps g(n)}"'] & M'' \arrow[r] \arrow[d, "\id_{M''}"]            & 0 \\
0 \arrow[r] & M' \arrow[r, "f"]                                                                 & M \arrow[r, "g"] \arrow[l, "?", dashed, bend left]                                                                     & M'' \arrow[r] \arrow[l, "?", dashed, bend left] & 0
\end{tikzcd}
            \]
            The diagram commutes. $\psi \circ f' = f, g \circ \psi = g'$, e.g: \[\psi \circ f'(m') = \psi(f(m'), 0) = f(m') + 0 = f(m')\] and
            \[g \circ \psi(b,n) = g(b+n) = \ubr{g(b)}_{=0} = g(n) = g(n) = g'(b,n)\]
            ($g(b) = 0$ is because $b \in \im(f) = \ker(g)$).

      \item For $s$: $f: M' \to \im(f)$ is an isomorphism ($f$ is injective) $\imp f^{-1}: \im(f) \to M'$ is an isomorphism. Check
            \[s = (f^{-1},0) \circ \psi^{-1}: \begin{tikzcd}[row sep = 30]
M \arrow[r, "\psi^{-1}"] & \im(f) \oplus N \arrow[r, "{\begin{matrix}(b,n) \mps f^{-1}(b) \\ \end{matrix}}"] & M'
\end{tikzcd}\]
      \item For $t$: Check that $s : N \to M''$ is an isomorphism using (b). Set $t:= i \circ g^{-1}$ for $i$ the inclusion so \[t: M'' \to N \hookrightarrow M\]Check. \qedhere
\end{itemize}
  \end{proof}
\end{thm-defi}
\begin{rem*}
  $M' \overunderset fs \rightleftarrows M $ and $M'' \overunderset tg \rightleftarrows M $ satisfy the condition from corollary 1.19, namely:
  \begin{itemize}
    \item $s \circ f = \id_{M'}$
    \item $g \circ t = \id_{M''}$
    \item $t \circ g + f \circ s = \id_{M}$
  \end{itemize}
  shows again: the sequence is split if $M \cong M' \oplus M''$ (for the ``right maps'')
\end{rem*}
\begin{rem}
  One also has short exact sequences for groups
  \[1 \to \ker(\pi) \overset s \leftrightarrows G \overunderset t \pi \leftrightarrows \bar G \to 1\]
  Here one has to be careful what splitting means. Having a $t$ is not equivalent to having an $s$.
  \[\ex t \iff G \cong \ker(\pi) \rtimes \bar G\]
  \[\ex s \iff G \cong \ker(\pi) \tm \bar G\]
\end{rem}

\section{Projective Modules}
\begin{defi}
  An $R$-module $P$ is called projective $\iff$ it has the following lifting property (LP; Hochhebungseigenschaft): In every diagram of $R$-modules
  \[\begin{tikzcd}
                   & P \arrow[ld, "\hat \varphi"', dashed] \arrow[d, "\varphi"] &   \\
M \arrow[r, "\pi"] & M' \arrow[r]                                               & 0
\end{tikzcd}\]
with $\pi$ surjective, there exists a lifting $\hat \ph : P \to M$ such that $\pi \circ \hat \ph = \ph$.
\end{defi}

\begin{prop}%31
\begin{enumerate}[(a)]
  \item Every free $R$-module is projective.
  \item For an $R$-module $P$ TFAE:
        \begin{enumerate}[(i)]
          \item $P$ is projective
          \item every s.e.s. $0 \to M' \to M \to P \to 0$ of $R$-modules splits.
                \item $P$ is a direct summand of a free module, i.e. $\ex R$-module $Q$, such that $P \oplus Q$ is a free $R$-module.
        \end{enumerate}
\end{enumerate}
\end{prop}
\begin{proof}
\begin{enumerate}[(a)]
  \item Let $P = R^{(I)}$ for a set $I$. Consider the diagram
        \[\begin{tikzcd}
                   & R^{(I)} \arrow[ld, "\hat \varphi"', dashed] \arrow[d, "\varphi"] &   \\
M \arrow[r, "\pi"] & M' \arrow[r]                                                     & 0
\end{tikzcd}\]

        Denote by $(e_{i})_{i \in I}$ the standard basis of $R^{(I)}$. $\ph$ is characterized by $m_{i}' := \ph(e_{i})$ for all $ i \in I$. (by universal property of $R^{(I)} = \bigoplus_{i \in I} R$). Because $\pi$ is surjective, we can choose a preimage $m_{i} \in M$ with $\pi(m_{i}) = m_{i}'$. Define $\hat \ph: R^{(I)} \to M$ as the unique $R$-module-homomorphism with $\hat \ph(e_{i}) = m_{i}$. Then $(\pi \circ \hat \ph)(e_{i}) = \pi(m_{i}) = m_{i}' = \ph(e_{i}) \imp \pi \circ \hat \ph = \ph.$
  \item
        \begin{itemize}
          \item (i)$\imp$(ii): Let $P$ be projective, consider a s.e.s.
                \[\begin{tikzcd}
0 \arrow[r] & M' \arrow[r] & M \arrow[r, "\pi"] & P \arrow[r]                                      & 0 \\
            &              &                    & P \arrow[u, "\id_P"'] \arrow[lu, "\psi", dashed] &
          \end{tikzcd}\]
                By the lifting property $\ex \psi: P \to M$ such that $\pi \circ \psi = \id_{P}$, i.e. $\psi$ is a splitting $\imp$ the s.e.s. splits.
          \item (ii)$\imp$(iii): From Remark 21 we have an $R$-module epimorphism $R^{(I)} \xrightarrow \pi P$ (for $I = P$). Take $Q:= \ker \pi$ ($\imp$ s.e.s. $0 \to Q \to R^{(I)} \to P \to 0$) By splitness $R^{(I)} = P \oplus Q$ (by Theorem 28).
          \item (iii)$\imp$(i): Start with a diagram
          % TODO
                \[\begin{tikzcd}
                   & P \arrow[d, "\ph"] &   \\
M \arrow[r, "\pi"] & M' \arrow[r]       & 0
\end{tikzcd}\]
                and assume $\ex R$-module $Q$ such that $P \oplus Q = R^{(I)}$ (really $\cong$). Extend $\ph$ to \[P \oplus Q \xrightarrow {\tilde \ph} M', (p,q) \mps \ph(p) + 0\]
                By (a) $\ex \hat{\tilde \ph} : P \oplus Q \to M$ with $\pi \circ \hat{\tilde \ph} = \tilde \ph$.
                \[\begin{tikzcd}
                   & P \oplus Q \arrow[d, "\tilde \ph"] \arrow[ld, "\hat{\tilde \ph}"', dashed] \\
M \arrow[r, "\pi"] & M'
\end{tikzcd}\]
                Set $\hat \ph:= \hat{\tilde \ph}|_{P \oplus Q}: P \to M$, check $\pi \circ \hat \ph = \ph$. \qedhere
        \end{itemize}
\end{enumerate}
\end{proof}

\begin{cor}%32
  Let $0 \to M' \to M \to M'' \to 0$ be a s.e.s. of $R$-modules.
  \begin{enumerate}[(a)]
    \item $M''$ projective $\imp$ ($M \cong M' \oplus M''$ and $M$ is projective $\iff M'$ projective).
    \item If $M'$ and $M''$ are free $R$-modules, then so is $M$. %make the union in the exponent next line a cupdot TODO
          \item If $M' \cong R^{(I)}$ and $M'' \cong R^{(I'')}$, then $M \cong R^{(I' \cup I'')}$. In particular, $rank(M) = rank(M') + rank(M'')$ if $I' \cup I''$ is finite.
  \end{enumerate}
\end{cor}
\begin{proof}
  \begin{enumerate}[(c)] %c b a order TODO
    \item clear: $R^{(I')} \oplus R^{(I'')} \cong R^{I'  \cup I''}$
    \item Follows from (a)
    \item
          \begin{itemize}
            \item First assertion in (a) $(M'' \imp M \cong M' \oplus M'')$ from Proposition 31.
            \item Second assertion: we know $M \cong M' \oplus M''$.
                  \item Suppose first: $M$ is projective. Then by 31(b)(iii): $\ex Q$ an $R$-module such that $M \oplus Q$
          \end{itemize}

\end{enumerate}

\end{proof}
\begin{thm}[Horse shoe lemma] Given a diagra of $R$-modules with $P', P''$ projective, and the first row exact
  \[\begin{tikzcd}
0 \arrow[r] & M' \arrow[r, "f"]  & M \arrow[r, "g"] & M'' \arrow[r]           & 0 \\
            & P' \arrow[u, "\a"] &                  & P'' \arrow[u, "\gamma"] &
\end{tikzcd}\]
\begin{enumerate}[(a)]
  \item The diagram can be completed by the dotted part to a commutative diagram, for a suitable $\beta: P' \oplus P'' \to M$, so that:
        \[\begin{tikzcd}
0 \arrow[r]         & M' \arrow[r, "f"]                                             & M \arrow[r, "g"]                                                                    & M'' \arrow[r]                             & 0 \\
0 \arrow[r, dashed] & P' \arrow[u, "\a"] \arrow[r, "{f': p' \mps (p',0)}"', dashed] & P' \oplus P'' \arrow[u, "\b", dashed] \arrow[r, "{g': (p',p'') \mps p''}"', dashed] & P'' \arrow[u, "\gamma"] \arrow[r, dashed] & 0
\end{tikzcd}\] and the second row is then also exact.
        \item If $\a$ and $\gamma$ are surjective, then so is $\b$.
\end{enumerate}
\end{thm}

\begin{proof}
\begin{enumerate}[(a)]
  \item Construction of $\b$: Use the lifting property of $P''$ to complete
        \[\begin{tikzcd}
M \arrow[r, "g"] & M'' \arrow[r]                                               & 0 \\
                 & P'' \arrow[u, "\gamma"'] \arrow[lu, "\hat \gamma ", dashed] &
               \end{tikzcd}\]
        By the diagonal arrow $\hat \gamma : P'' \to M$ to a commutative diagram. Define \[\b : P' \oplus P'' \to M, (x',x'') \mps f \circ \a(x') + \hat \gamma (x'')\]
        Check commutativities:
        \begin{itemize}
          \item $\b \circ f' \overset ? = f \circ \a$: \[\beta \circ f' (x') = \b(x',0) = f \circ \a(x') + \hat \gamma(0)\]
          \item $\gamma \circ g' \overset ?= g \circ \b$:
                \[g \circ \b(x',x'') = g(f \circ \a(x') + \hat \gamma(x'')) = TODO
                \]

        \end{itemize}
  \item Diagram chase:
        \[\begin{tikzcd}
0 \arrow[r] & M' \arrow[r, "f"]                      & M \arrow[r, "g"]                     & M'' \arrow[r]                      & 0 \\
0 \arrow[r] & P' \arrow[r, "f'"] \arrow[u, "\alpha"] & P \arrow[r, "g'"] \arrow[u, "\beta"] & P'' \arrow[u, "\gamma"'] \arrow[r] & 0
\end{tikzcd} TODO\]
        To show: $\b$ is surjective. Let $m \in M$, $\gamma$ surjective $\imp \ex x'' \in P'' : \gamma(x'') = g(m)$.
        $g'$ surjective $\imp \ex x \in P : g'(x) = x'$.

        Compare $m$ with $\b(x)$, consider $m - \b(x)$. Observe: $g(m-\b(x)) = g(m) - g \circ \b(x) = \gamma(x'') - g(x'') = 0$ TODO %Whole thing
\end{enumerate}
\end{proof}

\section{Finite generation, exact sequences and $\oplus$}
\begin{cor}
  Let $0 \to M' \xrightarrow f M \xrightarrow g M'' \to 0$ be a s.e.s. of $R$-modules, then
  \begin{enumerate}[(a)]
    \item If $M$ is finitely generated as $R$-module, then so is $M''$
    \item If $M'$ and $M''$ are finitely generated (as $R$-modules), then so is $M$.
  \end{enumerate}
\end{cor}
\begin{proof}
\begin{enumerate}[(a)]
  \item $M$ finitely generated $R$-module means $\ex$ finite set $I$ and $R$-module epimorphism $\pi: R^{(I)} \to M \imp R^{(I)} \to M''$ is given by $g \circ \pi$ is an epimorphism $\\imp M''$ is finitely generated as an $R$-module.
  \item Suppose we know $R$-module epimorphisms $\a: R^{(I')} \to M'$ and $\gamma: R^{(I'')} \to M''$, then Theorem 33 gives an $R$-module epimorphism \[\beta: R^{(I')} \oplus R^{(I'')} \to M\]
        $\imp M$ is finitely generated. \qedhere
\end{enumerate}
\end{proof}

\begin{rem*}
  $M$ is finitely generated as an $R$-module $\not \imp M' \le M$ is finitely generated. Example:
  let $R = M = K[X_{i} \mid i \in \N]$ and consider \[g: M \to K, X_{i} \mps 0, \forall i\]
  The kernel is the ideal $I$ of $R$ generated by $\{X_{i} \mid i \in \N\}$. We can check: $I$ is not a finitely generated $R$-module. If $I = (f_{1}, \ldots, f_{m})$ say $f_{1}, \ldots, f_{m} \in K[X_{1}, \ldots, X_{n}]$ TODO
\end{rem*}

\begin{cor}[exer]
  Let $M_{1}, \ldots, M_{n}$ be $R$-modules, then
  \begin{enumerate}[(a)]
    \item $M = \bigoplus_{1 \le i \le n}M_{i}$ is finitely generated $\iff M_{i}$ is finitely generated $\forall i$
          \item Suppose $M_{0} \subseteq \cdots \subseteq M_{n}$ with $\fak {M_{i}}{M_{i-1}}$ finitely generated for all $i \in \eb n$. Then $M_{n}$ is finitely generated.
  \end{enumerate}
\end{cor}

\begin{thm}[Snake lemma]
  Suppose we are given the following commutative diagram of $R$-modules with exact rows.
  \[\begin{tikzcd}
            & M' \arrow[r, "f"] \arrow[d, "\ph'"] & M \arrow[r, "g"] \arrow[d, "\ph"] & M'' \arrow[r] \arrow[d, "\ph''"] & 0 \\
0 \arrow[r] & N' \arrow[r, "f'"]                  & N \arrow[r, "g'"]                 & N''                              &
\end{tikzcd}\]
Then:
\begin{enumerate}[(a)]
  \item $\ex R$-linear map $\delta$ (called the connecting homomorphism) from $\ker(\ph'')$ to $\mathrm{coker}(\ph')$, such that the following sequence of $R$-modules is exact:
        TODO
  \item If $f$ is injective, then so is $f$.
        \item If $g'$ is surjective, then so is
\end{enumerate}
\end{thm}
\begin{proof}
  Construction of $\delta$: Given $m'' \in \ker \ph''$ map it to $m'' \in M''$ not $\ph''(m'') = 0$

\end{proof}

% Lecture 4

\begin{thm}[5-lemma]
  Suppose we are given the following commutative diagram of $R$-modules with exact rows
  \[\begin{tikzcd}
M_1 \arrow[r, "\alpha"] \arrow[d, "\varphi_1", two heads] & M_2 \arrow[r, "\beta"] \arrow[d, "\varphi_2"] & M_3 \arrow[r, "\gamma"] \arrow[d, "\varphi_3"] & M_4 \arrow[r, "\delta"] \arrow[d, "\varphi_4"] & M_5 \arrow[d, "\varphi_5", hook] \\
N_1 \arrow[r, "\alpha'"]                                  & N_2 \arrow[r, "\beta'"]                       & N_3 \arrow[r, "\gamma'"]                       & N_4 \arrow[r, "\delta'"]                       & N_5
\end{tikzcd}\]
and suppose that $\ph_{1}$ is surjective, $\ph_{5}$ is injective, and $\ph_{2}$ and $\ph_{4}$ are isomorphisms, then $\ph_{3}$ is also an isomorphism.
\end{thm}
\begin{proof} (in parts) Exercise.
  \begin{enumerate}
    \item version: diagram chase. %make sure break up into 3 TODO
          \item version: break up the diagram into 3 diagrams to which the snake lemma applies. \qedhere
  \end{enumerate}
\end{proof}

\section{Noetherian and Artinian modules and rings}

Let $R$ be a ring, $M, M', M'', M_i$ $R$-modules. A sequence $(M_{i})_{i \in \N}$ is said to \emph{become stationary} if $\ex i_{0} : \forall i \ge i_{0} : M_{i} = M_{i_{0}}$.
\begin{defi}
  $M$ is called
  \begin{enumerate}[(a)]
    \item noetherian$:\iff$ each ascending chain of submodules \[M_{0} \subseteq M_{1} \subseteq \cdots \subseteq M_{n} \subseteq \cdots \subseteq M\] becomes stationary (ACC for ascending chain condition)
    \item artinian$:\iff$ each descending chain of submodules \[M \supseteq M_0 \supseteq M_{1} \supseteq \cdots \supseteq M_{n} \supseteq \cdots\] becomes stationary (DCC for descending chain condition)
  \end{enumerate}
  and $R$ is called
  \begin{enumerate}[(a)]
          \setcounter{enumi}{2}
    \item left noetherian$:\iff$ it is noetherian as a left $R$-module
    \item left artinian$:\iff$ it is artinian as a left $R$-module
  \end{enumerate}
  analogously one defines right artinian/noetherian rings and modules.
\end{defi}

\begin{exmps}
  \begin{enumerate}[(a)]
    \item $\Z$ is noetherian but not artinian.
    \item Finite dimensional $K$-vector spaces are noetherian and artinian (use the dimension-function)
    \item (Exersize) Let $D$ be a skew field (division algebra), then any $D$-module is a free $D$-module. If a $D$-module is finitely generated, it is artinian and noetherian.\emph{In the present case one has a well-defined dimension for finitely generated $D$-modules}.
          \item Every field and every skew field is left and right artinian and noetherian. ($D\op$ is a skew field if $D$ is a skew field)
  \end{enumerate}
\end{exmps}

\begin{defi}
\begin{enumerate}[(a)]
  \item The center of $R$ is $Z(R) := \{r \in R \mid \forall r' \in R: r \cd r' = r' \cd r\}$, $Z(R)$ is a commutative subring (exersize)
  \item Let $S$ be any commutative ring and $\ph: S \to R$ be a ring homomorphism such that $\ph(S) \subseteq Z(R)$, then $R$ is called an $S$-algebra (via $\ph$).
\end{enumerate}
\end{defi}


\begin{exmps*}
  \begin{enumerate}[(a)]
  \item Every ring is a $\Z$-algebra (in a unique way)
    \item $K[X]$ is a $K$-algebra.
    \item If $R$ is finite dimensional $K$-algebra, then $R$ is left and right noetherian and artinian. (exercise) For instance, if $M$ is a finite monoid (or a finite group), then the monoid ring $K[M]$ is left and right artinian and noetherian.
          \item $S = \Q$-subalgebra of $2 \tm 2$ matrices over $\Q$ generated by $E = \begin{pmatrix} 1 & 0 \\ 0 & 1

          \end{pmatrix}$ and $\begin{pmatrix} 1 & 1 \\ 0 & 1
          \end{pmatrix} \imp S$ is commutative, but $MM_{2\tm 2}(\Q)$ over $S$ is not an $S$-algebra.
  \end{enumerate}
\end{exmps*}

\begin{facts}[Exercise; compare to linear algebra 2 (or Jantzen-Schwermer Ch. VIII.)]\item

  \begin{enumerate}[(a)]
    \item For $M$ the following are equivalent:
          \begin{enumerate}[(i)]
            \item Each subset of submodules of $M$ contains a maximal element.
                  \item Each submodule of $M$ is finitely generated.
          \end{enumerate}
    \item For $M$ the following are equivalent:
          \begin{enumerate}[(i)]
            \item $M$ is artinian
            \item Each subset of submodules of $M$ contains a minimal element.
          \end{enumerate}
  \end{enumerate}
\end{facts}
\begin{lemm}
  For submodules $N, P_{1}, P_{2} \subseteq M$ with
  \begin{enumerate}[(i)]
    \item $P_{1} \supseteq P_{2}$
    \item $P_{1} + N = P_{2} + N$
    \item $P_{1} \cap N = P_{2} \cap N$
  \end{enumerate}
  it follows that $P_{1} = P_{2}$.
\end{lemm}
\begin{proof}
  We need to show that $P_{1} \subseteq P_{2}$. Take $m_{1} \in P_{1} \underset{(ii)} \imp \ex m_{2} \in P_{2}, n \in N$ such that $m_{1} = m_{2} + n$. $\imp n = m_{1} - m_{2} \underset{P_{2} \subseteq P_{1}}\in P_{1} \cap N \underset{(iii)}= P_{2} \cap N$. $\imp m_{1} = m_{2} + \underbrace{n}_{\in P_{2} \cap N} \in  P_{2}$. \qedhere
\end{proof}

\begin{thm}
  Let $N \subseteq M$ be a submodule, then
  \begin{enumerate}[(a)]
    \item $M$ is noetherian (artinian) $\imp N$ and $\fak MN$ are noetherian (artinian).
          \item $N$ and $\fak MN$ are noetherian $\iff M$ is noetherian (artinian).
  \end{enumerate}
\end{thm}
\begin{proof}
\begin{enumerate}[(a)]
  \item For $N$: use directly the characterization (ii) from Facts 41. For $\fak MN$: use the homomorphism theorem to identify submodules of $\fak MN$ with those of $M$ containing $N$ and apply again (ii) from 41.
  \item Proof only in the artinian case: assume that $N$ and $\fak MN$ are artinian. Let $M \supseteq M_0 \supseteq M_{1} \supseteq \cdots \supseteq M_{n} \supseteq \cdots$ be a descending chain. Then by hypothesis: \[M \cap N\supseteq M_0 \cap N\supseteq M_{1} \cap N\supseteq \cdots \supseteq M_{n} \cap N \supseteq \cdots\]
        becomes stationary, as does \[M + N\supseteq M_0 + N\supseteq M_{1} + N\supseteq \cdots \supseteq M_{n} + N \supseteq \cdots\]
        $\imp \ex i_{0} : \forall i \ge i_{0}: M_{i} + N = M_{i_{0}} + N$ and $M_{i} \cap N = M_{i_{0}} \cap N$, we also have $M_{i} \subseteq M_{i_{0}}$, so by lemma 42 we have $M_{i} = M_{i_{0}} \imp (M_{i})$ becomes stationary. \qedhere
\end{enumerate}
\end{proof}

\begin{cor}[Exercise]\item
  \begin{enumerate}[(a)]
    \item Let $0 \to M' \to M \to M'' \to 0$ be a short exact sequence of $R$-modules. Then $M$ is noetherian (artinian) $\iff M'$  and $M''$ are noetherian (artinian).
          \item If $I$ is a finite set and $M := \bigoplus_{i \in I} M_{i}$, then $M$ is noetherian (artinian) $\iff$ all $M_{i}$ are noetherian (artinian).
  \end{enumerate}
Note: $R$ left-right noetherian (artinian) $\imp R^{n}$ is also left-right noetherian (artinian) $R$-module.
\end{cor}
\begin{cor}
  Let $R$ be left noetherian (artinian) and $M$ a finitely generated $R$-module, then $M$ is noetherian (artinian).
\end{cor}
\begin{proof}
$\ex$ an epimorphism $R^{n} \to M$. Now apply 44(a).\qedhere
\end{proof}
\begin{cor}
  Let $R$ be left noetherian (artinian) and $I \subseteq R$ a two-sided ideal, then the ring $\fak RI$ is also left noetherian (artinian).
\end{cor}
\begin{proof} $\fak RI$ is a ring (because $I$ is a two-sided ideal). $\fak RI$ is left noetherian (artinian) as an $R$-module by 44(a) $\imp \fak RI$ is left noetherian (artinian) as an $\fak RI$-module. \qedhere
\end{proof}
\begin{rem*}
$R$ is noetherian and $S \subseteq R$ a subring $\nimp S$ is noetherian because not every integral domain is noetherian, but its fraction field certainly is.
\end{rem*}

\begin{prop}
  Suppose we have $M \cong M \opl N$ for some $R$-module $N \ne 0$. Then $M$ is neither noetherian nor artinian.
\end{prop}
\begin{proof}[proof sketch]
  \begin{enumerate}
    \item $M \ne 0$ because $N \ne 0$ is a direct summand of it.
    \item $M \cong M \opl N \cong (M \opl N) \opl N \cong ((M \opl N) \opl N) \opl N \cong \cdots$.
  \end{enumerate}
  \begin{itemize}
    \item $\infty$ ascending chain:
          \[0 \opl N \subsetneq (0 \opl N) \opl N \subsetneq ((0 \opl N) \opl N) \opl N \subsetneq \cdots\]
          $\imp M$ is not noetherian.
    \item $\infty$ descending chain:
          \[M \supsetneq (M \opl 0) \supsetneq (M \opl 0) \opl 0 \supsetneq ((M \opl 0) \opl 0) \opl 0 \supsetneq \cdots\]
          $\imp M$ is not artinian. \qedhere
  \end{itemize}
\end{proof}

\begin{cor}[Exercise from 42 and 45]
  Suppose $R \ne 0$ is left noetherian (artinian), then for $n_{1}, n_{2} \in \Nn : R^{n_{1}} \cong R^{n_{2}} \imp n_{1} = n_{2}$ (In particular a rank of free finitely generated $R$-modules is defined.)
  \begin{proof}
    Assume $\ex n_{1}, n_{2} \in \Nn$ such that $R^{n_{1}} \cong R^{n_{2}}$, then $R^{n_1} \cong R^{n_{1}} \opl R^{n_{2} + n_{1}} \imp R^{n_{1}}$ not left noetherian (artinian). But 45 implies that $R^{n_1}$ is left noetherian (artinian) because $R$ has these properties. \qedhere
  \end{proof}
\end{cor}

\begin{thm}[Hilbert's basis theorem]
  If $R$ is left noetherian, then $R[X]$ is left noetherian (here $X$ commutes with elements of $R$).
  \begin{proof}
    TODO
  \end{proof}
\end{thm}
\section{Simple modules}
Let $R$ be a ring, $M, M', M'', M_{i}$ $R$-modules.
\begin{defi}
$M$ is called simple (or irreducibile) if $M \ne 0$ and $0$ and $M$ are the only $R$-submodules of $M$.
\end{defi}
\begin{exmps}

  \begin{enumerate}[(a)]
    \item Simple $K$-vector spaces are the 1-dimensional $K$-vectorspaces.
    \item Simple $\Z$-modules are $\fak \Z{p\Z}$ for $p$ a prime.
          \item A simple $M_{n \tm n}(K)$-module is $V_{n}(K)$ (space of column vectors).
  \end{enumerate}
\end{exmps}

\begin{defi}
  $M$ is said to have a composition series $\iff \ex$ finite descending chain of submodules \[M = M_{n} \supsetneq M_{n-1} \supsetneq \cdots \supsetneq M_{1} \supsetneq M_{0} = 0\]
  such that $\forall i \in \eb n$: the quotients $\fak{M_{i}}{M_{i-1}}$ are simple.
  The index $n$ is called the \emph{length} of $M$ and the quotients $\fak{M_{i}}{M_{i-1}}$ are called the \emph{factors} of $M$.
\end{defi}

\begin{prop}
$M$ has a decomposition series $\iff M$ is artinian and noetherian.
\begin{proof}
\begin{itemize}
  \item ``$\impliedby$'': Construct an ascending chain of submodules of $M$ as follows:
        \begin{align*}
          M_{0} &= 0 \\
          \rotatebox[origin=c]{270}{$\subseteq$}&\  \\
          M_{1} &= \text{a minimal submodule in } \{M' \le M \mid 0 \subsetneq M'\} \\
          \rotatebox[origin=c]{270}{$\subseteq$}&\  \\
          M_{2} &= \text{a minimal submodule in } \{M' \le M \mid M_{1} \subsetneq M'\}
          \rotatebox[origin=c]{270}{$\subseteq$}&\  \\
          \vdots  \\
        \end{align*}
        because $M$ is artinian, the $M_{i}$ exist (unless $M_{i+1} = M$), and because $M$ is noetherian, we will find an $n$ such that $M_{n} = M$. By the Homomorphism theorem: $\fak {M_{i}}{M_{i-1}}$ is simple $\forall i \in \eb n$.
  \item ``$\imp$'' : Suppose $M$ has a decomposition series and do induction on the minimal length of the series:
        \begin{itemize}
          \item $n = 1$: $M$ is simple.
                \item Induction step: Let $0 \subsetneq M_{1} \subsetneq \cdots \subsetneq M_{n+1} = M$, then $0 \subsetneq M_{1} \subsetneq \cdots \subsetneq M_{n}$ is a decomposition series of $M_{n}$ of minimal length $n$. (otherwise there would exist a shorter decomposition series of $M$ which would be a contradiction.) Induction hypothesis implies $M_{n}$ is artinian and noetherian, but also: $\fak M{M_{n}} = \fak {M_{n+1}}{M_{n}}$ is simple and hence artinian and noetherian. Consider $0 \to M_{n} \to M \to \fak M{M_{n}} \to 0 \underset{44}\imp M$ is artinian and noetherian. \qedhere
        \end{itemize}
\end{itemize}
\end{proof}
\end{prop}

\begin{thm}[Jordan-Hölder]
  Suppose $M$ has a decomposition series, then
  \begin{enumerate}[(a)]
    \item Any 2 decomposition series have the same length.
          \item The list of factors of $M$ (coming from any decomposition series) is unique up to  permutation (and isomorphism).
  \end{enumerate}
\end{thm}

\begin{defi*}
  Consider chains of submodules of $M$ (not necessarily decomposition series)
  \[\underline{\mathcal M}:= 0 \subsetneq M_{1} \subsetneq \cdots \subsetneq M_{n} = M\]
  \[\underline{\mathcal N}:= 0 \subsetneq N_{1} \subsetneq \cdots \subsetneq N_{n} = N\]
  \begin{enumerate}[(a)]
    \item Call $\underline{\mathcal N}$ a \emph{refinement} of $\underline{\mathcal M} :\iff \ebe{N_{1}}{N_{\ell}} \supseteq \ebe {M_{1}}{M_{n}}$.
    \item Call $\underline{\mathcal N}$ and $\underline{\mathcal M}$ \emph{equivalent} if $\ell = n$ and $\ex \s \in S_{k}$ such that $\forall i \in \eb k:$ \[\fak{N_{i}}{N_{i-1}} \cong \fak{M_{\s(i)}}{M_{\s(i)-1}}\]
  \end{enumerate}
\end{defi*}

\begin{lemm}[Schreier refinement lemma] [Jacobson Basic Algebra II, 3.6]%TODO Reference
  Any two finite length submodule chains possess equivalent refinements.
  \begin{proof}[Proof idea:] Use $\underline{\mathcal M}$ to refine each step $N_{i-1} \subsetneq N_{i}$.
    \begin{enumerate}[(1)]
\item $N_{i-1} \subseteq N_{i-1} + (M_{1} \cap N_{i}) \subseteq N_{i-1} + (M_{2} \cap N_{i}) \subseteq \cdots \subseteq N_{i-1} + (M_{k} \cap N_{i}) = N_{i}$.
\item Similarly $M_{j-1} \subseteq M_{j-1} + (N_{1} \cap M_{j}) \subseteq \cdots \subseteq M_{j-1} + (N_{ell} \cap M_{j}) = M_{j}$. Schreier verifies that the $j$-th subquotient of (1) and the $i$-th subquotient of (2) are isomorphic. (Butterfly lemma) \qedhere
    \end{enumerate}
  \end{proof}
\end{lemm}

\begin{proof}[Proof of Jordan-Hölder using Schreier refinement]
  Suppose $\underline{\mathcal M}$ and $\underline{\mathcal N}$ are decomposition series of $M$. Schreier refinement gives us refinements $\underline{\mathcal M}'$ of $\underline{\mathcal M}$ and $\underline{\mathcal N}$ of $\underline{\mathcal N}'$ such that $\underline{\mathcal M}'$ and $\underline{\mathcal N}'$ are equivalent, i.e. they have the same length and the same subfactors up to permutation and isomorphism. But $\underline{\mathcal M}$ and $\underline{\mathcal N}$ have no proper refinements $\imp \underline{\mathcal M}' = \underline{\mathcal M}$ and $\underline{\mathcal N}' = \underline{\mathcal N}$. \qedhere
\end{proof}
\begin{defi}
\begin{enumerate}[(a)]
  \item We say $M$ has finite length if $M$ is artinian and noetherian.
        \item If $M$ has finite length, then its length $\mathrm{len}(M)$ is the length of any decomposition series.
\end{enumerate}
\end{defi}

\begin{prop}
  Let $0 \to M' \xrightarrow \iota M \xrightarrow \pi M'' \to 0$ be a short exact sequence of $R$-modules, then:
  \begin{enumerate}[(a)]
    \item $M$ has finite length $\iff M'$ and $M''$ have finite length.
          \item $\mathrm{len}(M) = \mathrm{len}(M') + \mathrm{len}(M'')$.
  \end{enumerate}
\begin{proof}
\begin{enumerate}[(a)]
  \item See Cor. 44.
  \item Say $0 \subsetneq M_{1} \subsetneq \cdots \subsetneq M_{k} = M'$ and $0 \subsetneq \bar{M_{1}} \subsetneq \cdots \subsetneq \bar{M_{\ell}} = M''$ are decomposition series, then check (2nd isomorphism theorem) \[0 \subsetneq \iota(M_{1}) \subsetneq \cdots \subsetneq \iota(M_{k}) = \iota(M') = \pi^{-1}(0) \subsetneq \pi^{-1}(\bar M_{1}) \subsetneq \cdots \subsetneq\pi^{-1}(\bar M_{\ell}) = M\]
        is a decomposition series of $M$ of length $\ell + k = \mathrm{len}(M'') + \mathrm{len}(M')$. \qedhere
\end{enumerate}
\end{proof}
\end{prop}

\section{Indecomposable modules}
\begin{defi}
An $R$-module $M$ is indecomposable $\iff M \ne 0$ and there are no proper submodules $0 \subsetneq M_{1}, M_{2} \subsetneq M$ such that $M_{1} \opl M_{2} \xrightarrow \cong M, (m_{1},m_{2}) \mps m_{1}+m_{2}$.
\end{defi}
\begin{rem*}
\begin{enumerate}[(a)]
  \item If $M$ is simple, $M$ is indecomposable.
        \item If eveery submodule of finite length is a direct sum of simples submodules, then $M$ (indecomposable) of finite length $\imp M$ is simple.
\end{enumerate}
E.g. if $R$ is a field/skew field/$R = \Q[G]$ finite group, then indecomposable $\iff$ simple.
\end{rem*}
\begin{exmps*}
  \begin{itemize}
  \item Indecomposable $\Z$-modules are $\fak \Z{p^{n}\Z}$ for $p$ prime and $n \in\N$.
    \item All non-zero $\Z$-submodules of $\Q$ are indecomposable (e.g. $\Z, \Q$).
    \item Simple $\Z$-modules are $\fak \Z{p\Z}$ for $p$ prime.
          \item Indecomposable $K[X]$-modules are $\fak{K[X]}{(f^{n})}$ for $f \in K[X]$ irreducible and $n \in \N$ and $K[X]$-submodules of $K(X)$. (Any propoer submodule of $\fak{K[X]}{(f^{n})}$ is contained in $\fak{f K[X]}{(f^{n})}$)
  \end{itemize}
\end{exmps*}
\begin{thm}Suppose $M$ is noetherian or artinian, then $\ex n \in \N, \ebe {M_{1}}{M_{n}} \subseteq M$ indecomposable submodules such that \[M = \bigoplus_{1 \le i \le n} M_{i}.\]
  Call this statement $(*)_{M}$ for $M$. (A generalization of the structure theorem of finitely generated modules over a principle ideal domain; existence part onca indecomposable $R$-modules are understood).
  \begin{proof}(Assuming $M$ is artinian; other case is an exercise)
    Assume the statement of the theorem $(*)_{M}$ does not hold for $M$. Define $X = \{M' \le M \mid \neg (*)_{M'}\}$, then $X \ne \emptyset$ because $M \in X$ (by assumption). Let $M' \in X$ be a minimal element under $\subseteq$ (this existst since $M$ is artinian) , then $M'$ decomposable $\imp M' = M_{1} \opl M_{2}$ for proper submodules $0 \ne M_{1}, M_{2} \subsetneq M' \imp  M_{1}$ and $M_{2} \notin X \imp (*)_{M_{1}}$ and $(*)_{M_{2}}$ hold, i.e.
    \[M_{1} = \bigoplus_{1 \le i \le t}N_{i}, \quad M_{2} = \bigoplus_{1 \le j \le s}P_{j}\]where $N_{i}, P_{j}$ are indecomponible. \[\imp M_{1} \opl M_{2} = \bigoplus_{1 \le i \le t}N_{i} \opl \bigoplus_{1 \le j \le s}P_{j} \imp (*)_{M_{1} \opl M_{2}}.\]
    This is a contradiction to $M' \in X$ so $X$ must be empty.\qedhere
  \end{proof}
\end{thm}
\begin{rem*}
The decomposition in Theorem 59 is not unique
\end{rem*}
\begin{exer*}
  Let $M$ be a finitely generated module over a principle ideal domain, then $M$ is indecomposable $\iff M \cong R$ or $\ex n \in \N, \ex$ prime element $p \in R$ such that $M \cong \fak R{p^{n}}$.
  \begin{proof}Using structure theorem for modules over PIDs.
  \end{proof}
\end{exer*}

\begin{rem*}[Exercise]
  Recall: $e \in S$ a ring is called an idoempotent $\iff e^{2} = 2$ (nontrivial $\iff e \ne 0, 1$)
  $M$ is indecomposable $\iff 0_{M}, \id_{M} \in \End_{R}(M)$ are the only idempotents in $\End_{R}(M)$ (or else $M = e\cd M \opl (1-e)\cd M$).
\end{rem*}
\end{document}
